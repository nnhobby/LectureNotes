\chapter{Vector Calculus}

\section{Kronecker-Delta and Levi-Civita symbol}

\begin{equation}
\begin{split}
\epsilon_{ijk} \epsilon_{lmn} &=
   \begin{vmatrix}
   \delta_{il} & \delta_{im} & \delta_{in}  \\
   \delta_{jl} & \delta_{jm} & \delta_{jn}  \\
   \delta_{kl} & \delta_{km} & \delta_{kn}  \\
   \end{vmatrix} \\
&= \delta_{il} (\delta_{jm} \delta_{kn} - \delta_{jn} \delta_{km})
 - \delta_{im} (\delta_{jl} \delta_{kn} - \delta_{jn} \delta_{kl}) \\
&\quad + \delta_{in} (\delta_{jl} \delta_{km} - \delta_{jm} \delta_{kl})
\end{split}
\end{equation}

We use \textbf{Einstein sum convention},
then we can prove the following simpler and useful identities:
\begin{enumerate}[label=(\alph*)]
\item $\epsilon_{ijk} \epsilon_{imn} = \delta_{jm}\delta_{kn} - \delta_{jn}\delta_{km}$
\item $\epsilon_{ijk} \epsilon_{ijn} = 2 \delta_{kn}$
\end{enumerate}

\begin{proof}
(a)
\begin{equation*}
\begin{split}
\epsilon_{ijk} \epsilon_{imn}
&=  \epsilon_{ijk} \epsilon_{lmn} \delta_{il} \\
&= \delta_{il} \delta_{il} (\delta_{jm} \delta_{kn} - \delta_{jn} \delta_{km})
  - \delta_{il} \delta_{im} (\delta_{jl} \delta_{kn} - \delta_{jn} \delta_{kl}) \\
& \quad + \delta_{il} \delta_{in} (\delta_{jl} \delta_{km} - \delta_{jm} \delta_{kl}) \\
&= \delta_{ii} (\delta_{jm} \delta_{kn} - \delta_{jn} \delta_{km})
- \delta_{im} (\delta_{ji} \delta_{kn} - \delta_{jn} \delta_{ki}) \\
& \quad + \delta_{in} (\delta_{ij} \delta_{km} - \delta_{jm} \delta_{ki}) \\
&= 3 (\delta_{jm} \delta_{kn} - \delta_{jn} \delta_{km})
 - (\delta_{jm} \delta_{kn} - \delta_{jn} \delta_{km})
 + (\delta_{jn} \delta_{km} - \delta_{jm} \delta_{kn}) \\
 &= \delta_{jm}\delta_{kn} - \delta_{jn}\delta_{km}
\end{split}
\end{equation*}

(b)
\begin{equation*}
\begin{split}
\epsilon_{ijk} \epsilon_{ijn} &= \epsilon_{ijk} \epsilon_{imn} \delta_{jm} \\
&= \delta_{jm} (\delta_{jm}\delta_{kn} - \delta_{jn}\delta_{km}) \\
&= \delta_{jj} \delta_{kn} - \delta_{jn} \delta_{kj} \\
&= 2 \delta_{kn} \qedhere
\end{split}
\end{equation*}
\end{proof}

\section{Scalar and vector product}

\newcommand{\A}{\mathbf{A}}
\newcommand{\B}{\mathbf{B}}
\newcommand{\C}{\mathbf{C}}
\newcommand{\D}{\mathbf{D}}

\begin{definition}
The \textbf{scalar triple product} is defined by
$\A\cdot (\B\times\C)$ and the \textbf{vector triple product} is defined by
$\A\times (\B\times\C)$ where $\A, \B, \C\in \mathbf{R}^3$.
\end{definition}

\begin{property}
The scalar triple product has identities
\begin{equation}
\label{eq:scalar-triple-product}
\A\cdot (\B\times\C) = \B\cdot (\C\times\A) = \C\cdot (\A\times\B)
\end{equation}

The vector triple product can be expanded as
\begin{equation}
\label{eq:vector-triple-product}
\begin{split}
\A\times (\B\times\C) &= (\A\cdot\C)\B -(\A\cdot\B)\C \\
\B\times (\C\times\A) &= (\B\cdot\A)\C -(\B\cdot\C)\A \\
\C\times (\A\times\B) &= (\C\cdot\B)\A -(\C\cdot\A)\B
\end{split}
\end{equation}
\end{property}

These identities can be proved by using Kronecker-Delta and Levi-Civita symbol.
\begin{proof}
\Cref{eq:scalar-triple-product} can be expanded as following
\begin{equation*}
\begin{split}
\A\cdot (\B\times\C) &= A_i [\B\times\C]_i \\
&= A_i \epsilon_{ijk} B_j C_k \\
&\stackrel{i\leftrightarrow j}{=\joinrel=} B_i \epsilon_{jik} A_j C_k \\
&= B_i \epsilon_{ikj} C_k A_j \\
&= B_i [\C\times\A]_i \\
&= \B\cdot (\C\times\A)
\end{split}
\end{equation*}

The $i$-th component of \cref{eq:vector-triple-product} can be expanded as following
\begin{equation*}
\begin{split}
[\A\times(\B\times\C)]_i &= \epsilon_{ijk} A_j [\B\times\C]_k \\
&= \epsilon_{ijk} A_j \epsilon_{klm} B_l C_m \\
&= (\delta_{il}\delta_{jm} - \delta_{im}\delta_{jl}) A_j B_l C_m \\
&= A_j B_i C_j - A_j B_j C_i \\
&= [(\A\cdot\C)\B - (\A\cdot\B)\C]_i \qedhere
\end{split}
\end{equation*}
\end{proof}

Furthermore, we can get following identities,
\begin{enumerate}
\item Jacobi identity
\begin{equation}
\A\times(\B\times\C) + \B\times(\C\times\A) + \C\times(\A\times\B) = 0.
\end{equation}

\item scalar quadruple product
\begin{equation}
(\A\times\B)\cdot(\C\times\D) = (\A\cdot\C)(\B\cdot\D) - (\A\cdot\D)(\B\cdot\C)
\end{equation}

\item Lagrange’s identity
\begin{equation}
|\A\times\B|^2 = |\A|^2 |\B|^2 - (\A\cdot\B)^2
\end{equation}
\end{enumerate}

\section{Partial differentiation with non-independent variables}

\begin{example}
Let $w = x^2 + y^2 + z^2$ where $z = x^2 + y^2$. Calculate $\partial w/\partial x$.
There are two possible:

\begin{enumerate}[label = (\alph*)]
\item $x$, $y$ are independent
\begin{equation*}
\begin{split}
\frac{\partial w}{\partial x} &= \frac{\partial}{\partial x} (x^2 + y^2 + (x^2 + y^2)^2) \\
&= 2x + 2(x^2 + y^2)^2 \\
&= 4x^2 + 4y^2x + 2x.
\end{split}
\end{equation*}
\item $x$, $z$ are independent
\begin{equation*}
\begin{split}
\frac{\partial w}{\partial x} = \frac{\partial}{\partial x}(z + z^2) = 0.
\end{split}
\end{equation*}
\end{enumerate}

To sum up, the value of $\partial w/\partial x$ depends on which variables are independent.
To avoid the ambiguity, we should using the following notation:

\begin{enumerate}[label=Case (\alph*), wide, itemsep =0pt]
\item  $x$, $y$ are the independent variables: $\left(\frac{\partial w}{\partial x}\right)_y$
\item  $x$, $z$ are the independent variables: $\left(\frac{\partial w}{\partial x}\right)_z$.
\end{enumerate}

These are read, “the partial of $w$ with respect to $x$, with $y$ (resp. $z$) held constant”.
\end{example}

\begin{example}
\begin{equation*}
  w = f(x, y, z, t), \quad \textrm{where}\quad xy=zt,
\end{equation*}
then only three of the variables $x$, $y$, $z$, $t$ can be independent.
Thus we would write expressions like

\begin{enumerate}[label=(\alph*), wide, itemsep =0pt]
\item  $x$, $y$ are the independent variables: $\left(\frac{\partial w}{\partial x}\right)_y$
\item  $x$, $z$ are the independent variables: $\left(\frac{\partial w}{\partial x}\right)_z$.
\end{enumerate}

\end{example}
