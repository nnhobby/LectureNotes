\part{Real Analysis}

\chapter{sequences and series}

\textbf{theorem vs proposition vs lemma} \newline

The main results are theorems, smaller results are called propositions.
A Lemma is a technical intermediate step which has no standing as an independent result.
Lemmas are only used to chop big proofs into handy pieces.

\section{sequence}

\begin{theorem}
A sequence $(a_n)^{\infty}_{n=1}$ of real numbers is convergent if and only if (iff) it is a Cauchy sequence.
\end{theorem}

\begin{proposition}[Monotone bounded sequences converge]
Let $(a_n)^\infty_{n=m}$ be a sequence of real numbers which has some
finite upper bound $M \in \mathbb{R} $, and which is also increasing (i.e., $a_{n+1} \ge a_n$ for all $n \ge m$).
Then $(a_n)^\infty_{n=m}$ is convergent, and in fact
\begin{equation*}
\lim_{n \to \infty} a_n = sup(a_n)^\infty_{n=m} \le M
\end{equation*}
\end{proposition}