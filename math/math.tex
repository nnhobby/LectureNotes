\documentclass[a4paper,twoside,12pt]{book}
\usepackage[T1]{fontenc}
%% set input font, after the 2018 release of LaTeX, default is utf8
\usepackage[utf8]{inputenc}
\usepackage{lmodern}
\usepackage{xcolor}

% use fancyhdr to customize page headers
% ref: https://tex.stackexchange.com/questions/366049/heading-not-in-capital-letter
\usepackage{fancyhdr}
\pagestyle{fancy}
\fancyhf{}
\fancyhead[LE,RO]{\thepage}
\fancyhead[LO]{\itshape\nouppercase{\rightmark}}
\fancyhead[RE]{\itshape\nouppercase{\leftmark}}
\renewcommand{\headrulewidth}{0pt}
\setlength{\headheight}{14.5pt}

% Adding package bookmark improves bookmarks handling.
% More features and faster updated bookmarks.
\usepackage{graphicx}
\usepackage[english]{babel}
\usepackage{enumitem}
\usepackage{listings}

%% math
\usepackage{amsmath}
\usepackage{amssymb}
\usepackage{amsthm} % must be loaded after amsmath package
\usepackage{thmtools}

% 1. set different style for theorem like, definition like, remark like
% 2. set lemma, proposition, definition, etc. are all share same counter,
%        so, they are numbered sequentially.
%
% style explains:
% plain: Theorem, Lemma, Corollary, Proposition, Conjecture,
%        Criterion, Assertion
%
% definition : Definition, Condition, Problem, Example, Exercise,
%              Algorithm, Question, Axiom, Property, Assumption,
%              Hypothesis
%
% remark : Remark, Note, Notation, Claim, Summary,
%          Acknowledgment, Case, Conclusion

\theoremstyle{plain}
   \newtheorem{theorem}{Theorem}[section]
   \newtheorem{lemma}[theorem]{Lemma}
   \newtheorem{corollary}[theorem]{Corollary}
   \newtheorem{proposition}[theorem]{Proposition}
\theoremstyle{definition}
   \newtheorem{definition}[theorem]{Definition}
   \newtheorem{property}[theorem]{Property}
   \newtheorem{example}[theorem]{Example}
\theoremstyle{remark}
  \newtheorem{remark}[theorem]{Remark}

%%%%%%%%%%%%%%%%%%%%%%%%%%%%%%%%%%%%%%%%%%%%%%%%
% Chapter quote at the start of chapter        %
% Source: http://tex.stackexchange.com/a/53380 %
%%%%%%%%%%%%%%%%%%%%%%%%%%%%%%%%%%%%%%%%%%%%%%%%
\usepackage{epigraph}

%%%%%%%%%%%%%%%%%%%%%%%%%%%%%%%%%%%%%%%%%%%%%%%%%%%%%%%%%
% Source: http://en.wikibooks.org/wiki/LaTeX/Hyperlinks %
%%%%%%%%%%%%%%%%%%%%%%%%%%%%%%%%%%%%%%%%%%%%%%%%%%%%%%%%%
\usepackage[bookmarksopen=true,bookmarksnumbered]{hyperref}
\usepackage[noabbrev, nameinlink]{cleveref} % to be loaded after hyperref
\usepackage[openlevel=0]{bookmark}

%%% Book's title and subtitle
\title{\Huge \textbf{Maths Notes}}
\author{\textsc{nnhobby}}
\date{} % remove date

\begin{document}

%%%%%%%%%%%%%%%%%%%%%%%%%%%%%%%%%%%%%%%%%%
%%   book order                         %%
%%%%%%%%%%%%%%%%%%%%%%%%%%%%%%%%%%%%%%%%%%
% ref: https://tex.stackexchange.com/questions/20538/what-is-the-right-order-when-using-frontmatter-tableofcontents-mainmatter/20541
% 1. \frontmatter turns off chapter numbering and uses roman numerals for page numbers;
% 2. \mainmatter turns on chapter numbering, resets page numbering
%                and uses arabic numerals for page numbers;
% 3. \appendix resets chapter numbering, uses letters for chapter numbers
%              and doesn't fiddle with page numbering;
% 4. \backmatter turns off chapter numbering and doesn't fiddle with page numbering.
\frontmatter
\maketitle

%%%%%%%%%%%%%%%%%%%%%%%%%%%%%%%%%%%%%%%%%%%%%%%%%%%%%%%%%%%%%%%%%%%%%%%%
% Auto-generated table of contents, list of figures and list of tables %
%%%%%%%%%%%%%%%%%%%%%%%%%%%%%%%%%%%%%%%%%%%%%%%%%%%%%%%%%%%%%%%%%%%%%%%%
% add toc to bookmark, \cleardoublepage avoid to link blank page!
\cleardoublepage
\pdfbookmark{\contentsname}{toc}
\tableofcontents

% notations
\chapter*{Notations}
\label{notation}

% Sometimes we have to include the following line to get this section
% included in the Table of Contents despite being a chapter*
\addcontentsline{toc}{chapter}{Notations}

\clearpage
\listoffigures
\listoftables

% turn on chapter numbering to arabic number
\mainmatter

% use \input over \include
% ref: https://tex.stackexchange.com/questions/20538/what-is-the-right-order-when-using-frontmatter-tableofcontents-mainmatter/20541#20541
\part{Real Analysis}

\chapter{sequences and series}

\textbf{theorem vs proposition vs lemma} \newline

The main results are theorems, smaller results are called propositions.
A Lemma is a technical intermediate step which has no standing as an independent result.
Lemmas are only used to chop big proofs into handy pieces.

\section{sequence}

\begin{theorem}
A sequence $(a_n)^{\infty}_{n=1}$ of real numbers is convergent if and only if (iff) it is a Cauchy sequence.
\end{theorem}

\begin{proposition}[Monotone bounded sequences converge]
Let $(a_n)^\infty_{n=m}$ be a sequence of real numbers which has some
finite upper bound $M \in \mathbb{R} $, and which is also increasing (i.e., $a_{n+1} \ge a_n$ for all $n \ge m$).
Then $(a_n)^\infty_{n=m}$ is convergent, and in fact
\begin{equation*}
\lim_{n \to \infty} a_n = sup(a_n)^\infty_{n=m} \le M
\end{equation*}
\end{proposition}
\part{Functional Analysis}

\chapter{Metric Space}

\begin{definition}[Metric space]
A Metric space is a pair $(X, d)$, where $X$ is a set and $d$ is a distance function or metric on $X$ such that $d: X\times X \to [0, +\infty)$.
Furthermore, the metric must satisfy the following four axioms:
\begin{enumerate}[label=(\alph*)]
\item  $d(x, y) = 0$ if and only if $x = y$
\item  (Positivity) for any distinct $x, y \in X, d(x, y) > 0$ 
\item  (Symmetry) $d(x, y) = d(y, x)$ 
\item  (Triangle inequality) $d(x, y) <= d(x, z) + d(z, y)$ 
\end{enumerate}
\end{definition}
\chapter{Calculus of Variations}

\section{Euler-Lagrange Equation}

\begin{lemma}
\label{lemma:variation-1}
If $f(x) \in C[a,b]$, and if
\begin{equation*}
\int_a^b f(x) g(x) \,dx = 0
\end{equation*}
for $\forall g(x) \in L(a,b)$ such that $g(a) = g(b) = 0$, 
then $f(x) = 0, \forall x \in [a, b]$.
\end{lemma}

\begin{proof}
Suppose $f(x) > 0$ in some $[x_1, x_2]$, if we set $g(x) = (x - x_1)(x_2 - x)$, then
\begin{equation*}
\int_{x_1}^{x_2} f(x)(x - x_1)(x_2 - x)\,dx > 0 \,,
\end{equation*}

since the integrand is positive. This contradiction proves the lemma.
\end{proof}

\begin{remark}
The \autoref{lemma:variation-1} still holds if we replace linear space $L(a,b)$ 
by normed linear space $N_n(a,b)$. To proof this, just set
\begin{equation*}
    g(x) = \left[ (x - x_1)(x_2 - x) \right]^n
\end{equation*}
\end{remark}


Problem: find the extremum of \cref{eq:variation-1}.

\begin{equation}
    \label{eq:variation-1}
    J[y] = \int_{a}^{b} F[x, y, y']\, dx
\end{equation}


Solution 1:

\begin{equation*}
\begin{split}
0 & = \delta J[y] \\
  & = \int_a^b \left[F(x, y + \delta y, y' + \delta y') - F(x, y, y')\right] \, dx\\ 
  & = \int_a^b \left[\frac{\partial F}{\partial y} \delta y 
      + \frac{\partial F}{\partial y'} \delta y' \right]\,dx\\
  & = \int_a^b \left[ \frac{\partial F}{\partial y} \delta y
      - \frac{d}{dx} \left( \frac{\partial F}{\partial y'}\right) \delta y \right] dx 
      + \left.\frac{\partial F}{\partial y'} \delta y \right\vert_a^b \\
  & = \int_a^b \left[ \frac{\partial F}{\partial y}
      - \frac{d}{dx} \left( \frac{\partial F}{\partial y'}\right) \right]\delta y \,dx 
\end{split}
\end{equation*}

So, 
\begin{equation}
\label{Euler-Lagrange-equation}
\frac{\partial F}{\partial y} - \frac{d}{dx} \left( \frac{\partial F}{\partial y'}\right) = 0
\end{equation}
which is called Euler-Lagrange equation.

\part{Ordinary Differential Equations}

\chapter{Brachistochrone Problem}

\chapter{First Order Equations}

\section{Homogeneous Equations}

\begin{equation*}
    \frac{dy}{dx} = f(x, y)
\end{equation*}
\part{Numerical Methods}

\chapter{Coursera: Numerical Methods for Engineers}

\section{Logistic Map}

Mathematically, the logistic map is written
\begin{equation}
    \label{eq:logistic_map}
    x_{n+1} = r x_n (1 - x_n),
\end{equation}
where $r$ (sometimes also denoted $\mu$) is a positive constant.

\subsection{Fixed Point}
The fixed point means that some point satisfies $x = f(x)$. By solving 
\begin{equation*}
    x = rx(1-x), \quad x = 0 \quad \text{or}\quad  x = 1 - \frac{1}{r}
\end{equation*}


\subsection{Period-2 Point}
We say that $x_1$ and $x_2$ are a period-2 cycle of a one-dimensional map $f(x)$ if
\begin{equation*}
x_2 = f(x_1) \quad\textrm{and}\quad  x_1 = f(x_2) \quad\textrm{and}\quad x_1 \ne x_2 . 
\end{equation*}
Determine the period-2 cycle for the logistic map by solving the equation $x = f(f(x))$,
with $f(x) = rx(1-x)$.

\input{topics/numerical_methods/linear_algebra}
\part{Others To Be Orgnized}

\chapter{Vector Calculus}

\section{Kronecker-Delta and Levi-Civita symbol}

\begin{equation}
\begin{split}
\epsilon_{ijk} \epsilon_{lmn} &=
   \begin{vmatrix}
   \delta_{il} & \delta_{im} & \delta_{in}  \\
   \delta_{jl} & \delta_{jm} & \delta_{jn}  \\
   \delta_{kl} & \delta_{km} & \delta_{kn}  \\
   \end{vmatrix} \\
&= \delta_{il} (\delta_{jm} \delta_{kn} - \delta_{jn} \delta_{km})
 - \delta_{im} (\delta_{jl} \delta_{kn} - \delta_{jn} \delta_{kl}) \\
&\quad + \delta_{in} (\delta_{jl} \delta_{km} - \delta_{jm} \delta_{kl})
\end{split}
\end{equation}

We use \textbf{Einstein sum convention},
then we can prove the following simpler and useful identities:
\begin{enumerate}[label=(\alph*)]
\item $\epsilon_{ijk} \epsilon_{imn} = \delta_{jm}\delta_{kn} - \delta_{jn}\delta_{km}$
\item $\epsilon_{ijk} \epsilon_{ijn} = 2 \delta_{kn}$
\end{enumerate}

\begin{proof}
(a)
\begin{equation*}
\begin{split}
\epsilon_{ijk} \epsilon_{imn}
&=  \epsilon_{ijk} \epsilon_{lmn} \delta_{il} \\
&= \delta_{il} \delta_{il} (\delta_{jm} \delta_{kn} - \delta_{jn} \delta_{km})
  - \delta_{il} \delta_{im} (\delta_{jl} \delta_{kn} - \delta_{jn} \delta_{kl}) \\
& \quad + \delta_{il} \delta_{in} (\delta_{jl} \delta_{km} - \delta_{jm} \delta_{kl}) \\
&= \delta_{ii} (\delta_{jm} \delta_{kn} - \delta_{jn} \delta_{km})
- \delta_{im} (\delta_{ji} \delta_{kn} - \delta_{jn} \delta_{ki}) \\
& \quad + \delta_{in} (\delta_{ij} \delta_{km} - \delta_{jm} \delta_{ki}) \\
&= 3 (\delta_{jm} \delta_{kn} - \delta_{jn} \delta_{km})
 - (\delta_{jm} \delta_{kn} - \delta_{jn} \delta_{km})
 + (\delta_{jn} \delta_{km} - \delta_{jm} \delta_{kn}) \\
 &= \delta_{jm}\delta_{kn} - \delta_{jn}\delta_{km}
\end{split}
\end{equation*}

(b)
\begin{equation*}
\begin{split}
\epsilon_{ijk} \epsilon_{ijn} &= \epsilon_{ijk} \epsilon_{imn} \delta_{jm} \\
&= \delta_{jm} (\delta_{jm}\delta_{kn} - \delta_{jn}\delta_{km}) \\
&= \delta_{jj} \delta_{kn} - \delta_{jn} \delta_{kj} \\
&= 2 \delta_{kn} \qedhere
\end{split}
\end{equation*}
\end{proof}

\section{Scalar and vector product}

\newcommand{\A}{\mathbf{A}}
\newcommand{\B}{\mathbf{B}}
\newcommand{\C}{\mathbf{C}}
\newcommand{\D}{\mathbf{D}}

\begin{definition}
The \textbf{scalar triple product} is defined by
$\A\cdot (\B\times\C)$ and the \textbf{vector triple product} is defined by
$\A\times (\B\times\C)$ where $\A, \B, \C\in \mathbf{R}^3$.
\end{definition}

\begin{property}
The scalar triple product has identities
\begin{equation}
\label{eq:scalar-triple-product}
\A\cdot (\B\times\C) = \B\cdot (\C\times\A) = \C\cdot (\A\times\B)
\end{equation}

The vector triple product can be expanded as
\begin{equation}
\label{eq:vector-triple-product}
\begin{split}
\A\times (\B\times\C) &= (\A\cdot\C)\B -(\A\cdot\B)\C \\
\B\times (\C\times\A) &= (\B\cdot\A)\C -(\B\cdot\C)\A \\
\C\times (\A\times\B) &= (\C\cdot\B)\A -(\C\cdot\A)\B
\end{split}
\end{equation}
\end{property}

These identities can be proved by using Kronecker-Delta and Levi-Civita symbol.
\begin{proof}
\Cref{eq:scalar-triple-product} can be expanded as following
\begin{equation*}
\begin{split}
\A\cdot (\B\times\C) &= A_i [\B\times\C]_i \\
&= A_i \epsilon_{ijk} B_j C_k \\
&\stackrel{i\leftrightarrow j}{=\joinrel=} B_i \epsilon_{jik} A_j C_k \\
&= B_i \epsilon_{ikj} C_k A_j \\
&= B_i [\C\times\A]_i \\
&= \B\cdot (\C\times\A)
\end{split}
\end{equation*}

The $i$-th component of \cref{eq:vector-triple-product} can be expanded as following
\begin{equation*}
\begin{split}
[\A\times(\B\times\C)]_i &= \epsilon_{ijk} A_j [\B\times\C]_k \\
&= \epsilon_{ijk} A_j \epsilon_{klm} B_l C_m \\
&= (\delta_{il}\delta_{jm} - \delta_{im}\delta_{jl}) A_j B_l C_m \\
&= A_j B_i C_j - A_j B_j C_i \\
&= [(\A\cdot\C)\B - (\A\cdot\B)\C]_i \qedhere
\end{split}
\end{equation*}
\end{proof}

Furthermore, we can get following identities,
\begin{enumerate}
\item Jacobi identity
\begin{equation}
\A\times(\B\times\C) + \B\times(\C\times\A) + \C\times(\A\times\B) = 0.
\end{equation}

\item scalar quadruple product
\begin{equation}
(\A\times\B)\cdot(\C\times\D) = (\A\cdot\C)(\B\cdot\D) - (\A\cdot\D)(\B\cdot\C)
\end{equation}

\item Lagrange’s identity
\begin{equation}
|\A\times\B|^2 = |\A|^2 |\B|^2 - (\A\cdot\B)^2
\end{equation}
\end{enumerate}

\section{Partial differentiation with non-independent variables}

\begin{example}
Let $w = x^2 + y^2 + z^2$ where $z = x^2 + y^2$. Calculate $\partial w/\partial x$.
There are two possible:

\begin{enumerate}[label = (\alph*)]
\item $x$, $y$ are independent
\begin{equation*}
\begin{split}
\frac{\partial w}{\partial x} &= \frac{\partial}{\partial x} (x^2 + y^2 + (x^2 + y^2)^2) \\
&= 2x + 2(x^2 + y^2)^2 \\
&= 4x^2 + 4y^2x + 2x.
\end{split}
\end{equation*}
\item $x$, $z$ are independent
\begin{equation*}
\begin{split}
\frac{\partial w}{\partial x} = \frac{\partial}{\partial x}(z + z^2) = 0.
\end{split}
\end{equation*}
\end{enumerate}

To sum up, the value of $\partial w/\partial x$ depends on which variables are independent.
To avoid the ambiguity, we should using the following notation:

\begin{enumerate}[label=Case (\alph*), wide, itemsep =0pt]
\item  $x$, $y$ are the independent variables: $\left(\frac{\partial w}{\partial x}\right)_y$
\item  $x$, $z$ are the independent variables: $\left(\frac{\partial w}{\partial x}\right)_z$.
\end{enumerate}

These are read, “the partial of $w$ with respect to $x$, with $y$ (resp. $z$) held constant”.
\end{example}

\begin{example}
\begin{equation*}
  w = f(x, y, z, t), \quad \textrm{where}\quad xy=zt,
\end{equation*}
then only three of the variables $x$, $y$, $z$, $t$ can be independent.
Thus we would write expressions like

\begin{enumerate}[label=(\alph*), wide, itemsep =0pt]
\item  $x$, $y$ are the independent variables: $\left(\frac{\partial w}{\partial x}\right)_y$
\item  $x$, $z$ are the independent variables: $\left(\frac{\partial w}{\partial x}\right)_z$.
\end{enumerate}

\end{example}


\cleardoublepage
\phantomsection
\appendix
\addcontentsline{toc}{part}{Appendices}

\chapter{\LaTeX{} Notes}

\section{\LaTeX{} list: enumitem}

List structures in \LaTeX{} have three types:

\begin{itemize}
\item \textcolor{blue}{itemize} for a bullet list
\item \textcolor{blue}{enumerate} for an enumerated list
\item \textcolor{blue}{description} for a descriptive list.
\end{itemize}

The description environment can have optional label:
\begin{verbatim}
    \item[label text] Text of your description goes here...
\end{verbatim}

\subsection{Customizing lists}


\section{How to produce png file?}

Fist use \textit{standalone} package with option \textbf{preview}:

\begin{verbatim}
\documentclass[preview, convert={outext=.png}]{standalone}
\end{verbatim}

Then compile tex source code:

\begin{verbatim}
latexmk -shell-escape textfile
\end{verbatim}


\end{document}
