\documentclass[a4paper,12pt]{article}
\usepackage[T1]{fontenc}
%% set input font, after the 2018 release of LaTeX, default is utf8
\usepackage[utf8]{inputenc}
\usepackage{lmodern}
\usepackage{xcolor}
\usepackage{hyperref}

%% math
\usepackage{amsmath}
\usepackage{amssymb}
\usepackage{amsthm} % must be loaded after amsmath package
\usepackage{thmtools}
\usepackage{algorithm}
\usepackage{algorithmic}

\title{General Matrix Multiplication}
\author{nnhobby}
\date{}

\begin{document}
\maketitle

\section{Loop order}
Matrix-matrix multiply is denoted as $C := AB + C$, and $m, n, k$ denotes the matrix size.
So, the loop index is denoted by $i, j, p$.

\begin{align*}
&\textbf{for~} i := 0, \ldots , m-1 \\
&\hspace{1em} \textbf{for~} j := 0, \ldots , n-1 \\
&\hspace{2em} \textbf{for~} p := 0, \ldots , k-1 \\
&\hspace{3em} \gamma_{i,j} := \alpha_{i,p} \beta_{p,j} + \gamma_{i,j} \\
&\hspace{2em} \textbf{end} \\
&\hspace{1em} \textbf{end} \\
&\textbf{end}
\end{align*}


\begin{center}
\begin{tabular}{|l|c|}
\hline
{$\!\begin{aligned}
&\textbf{for~} i := 0, \ldots , m-1 \\
&\hspace{1em} \textbf{for~} j := 0, \ldots , n-1 \\
&\hspace{2em} \textbf{for~} p := 0, \ldots , k-1 \\
&\hspace{3em} \gamma_{i,j} := \alpha_{i,p} \beta_{p,j} + \gamma_{i,j} \\
&\hspace{2em} \textbf{end} \\
&\hspace{1em} \textbf{end} \\
&\textbf{end}
\end{aligned}$}
& figure \\
\hline
\end{tabular}
\end{center}


\begingroup
\arraycolsep=1.4pt\def\arraystretch{1.5}
\begin{equation*}
\begin{array}{l}
\textbf{for~}  j := 0, \ldots , n-1 \\
\hspace{1em}
\left. \begin{array}{l}
{\bf for~} p := 0, \ldots , k-1  \\
\hspace{1em}
\left. \begin{array}{l}
\textbf{for~} i := 0, \ldots , m-1  \\
\hspace{1em} \gamma_{i,j} := \alpha_{i,p} \beta_{p,j} + \gamma_{i,j}  \\
\textbf{end}
\end{array} \right\} \hspace{1em} \begin{array}[t]{c}
\underbrace{c_j := \beta_{p,j} a_p  + c_j}\\
\mbox{axpy}
\end{array}
\\
\textbf{end}
\end{array} \right\} \hspace{1em} \begin{array}[t]{c}
\underbrace{c_j := A b_j + c_j}\\
\mbox{mv mult}
\end{array}
\\
\textbf{end}
\end{array}
\end{equation*}
\endgroup


\appendix

\section{latex}

\url{https://tex.stackexchange.com/questions/121407/align-inside-of-tabular}


\begin{verbatim}
When to use \begingroup?
\begingroup is a TeX primitive for group.
\end{verbatim}


\end{document}
